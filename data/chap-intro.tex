\chapter{Regular expression}
\label{chap:intro}

In \textsf{theoretical computer science} and \textsf{formal language
  theory}, a \textbf{regular expression} (abbreviated \textbf{regex}
or \textbf{regexp} and sometimes called a \textbf{rational
  expression}) is a sequence of characters that define a search
pattern, mainly for use in pattern matching with strings, or string
matching, i.e. ``find and replace''-like operations. The concept arose
in the 1950s, when the American mathematician Stephen Kleene
formalized the description of a \textsl{regular language}, and came
into common use with the Unix text procession utilities \verb|ed|, an
editor, and \verb|grep| (global regular expression print), a filter.

Regular expressions are so useful in computing that the various
systems to specify regular expressions have evolved to provide both a
\textsl{basic} and \textsl{extended} standard for the grammar and
syntax; \textsl{modern} regular expressions heavily augment the
standard. Regular expression processors are found in several search
engines, search and replace dialogs of several word processors and
text editors, and in the command line of text processing utilities,
such as \verb|sed| and \verb|AWK|.

Many programming languages provide regular expression capabilities,
some built-in, for example Perl, JavaScript, Ruby, AWK, and Tcl, and
others via a standard library, for example .NET languages, Java,
Python and C++ (since C++11). Most other languages offer regular
expressions via a library.
