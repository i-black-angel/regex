\chapter{Basic concepts}
\label{chap:concept}

A regular expression, often called a \textbf{pattern}, is an
expression used to specify a set of strings required for a particular
purpose. A simple way to specify a finite set of strings is to list
its elements or members. However, there are often more concise ways to
specify the desired set of strings. For example, the set containing
the three strings ``Handel'', ``H\"andel'', and ``Haendel'' can be
specified by the \textbf{pattern} \texttt{H(\"a|ae?)ndel}; we say that
this pattern \textbf{matches} each of the three strings. In most
formalisms, if there exists at least one regex that matches a
particular set then there exists an infinite number of other regex
that also match it---the specification is not unique. Most formalisms
provide the following operations to construct regular expressions.

\section{Boolean ``or''}
  A vertical bar separates alternatives. For example, \verb/gray|grey/ can
  match ``gray'' or ``grey''.

\section{Grouping}
  Parentheses are used to define the scope and precedence of the
  operators (among other uses). For example, \verb/gray|grey/ and
  \verb/gr(a|e)y/ are equivalent patterns which both describe the set
  of ``gray'' or ``grey''.

\section{Quantification}
  A quantifier after a token (such as a character) or group specifies
  how often that preceding element is allowed to occur. The most
  common quantifiers are the question mark \verb|?|, the asterisk
  \verb|*| (derived from the Kleene star), and the plus sign \verb|+| (Kleene
  plus).

  \begin{tabularx}{335pt}{lX}
    \verb|?| & The question mark indicates there is \textsl{zero or
    one} of the preceding element. For example, \verb|colou?r| matches
    both ``color'' and ``colour''.\\
    \verb|*| & The asterisk indicates there is \textsl{zero or
    more} of the preceding element. For example, \verb|ab*c| matches
    ``ac'', ``abc'', ``abbc'', ``abbbc'', and so on.\\
    \verb|+| & The plus sign indicates there is \textsl{one or
    more} of the preceding element. For example, \verb|ab+c| matches
    ``abc'', ``abbc'', ``abbbc'', and so on, but not ``ac''.
  \end{tabularx}

  These constructions can be combined to form arbitrarily complex
  expressions, much like one can construct arithmetical expressions
  from numbers and the operations $+$, $-$, $\times$, and $\div$. For
  example, \texttt{H(ae?|\"a)ndel} and \texttt{H(a|ae|\"a)ndel} are
  both valid patterns which match the same strings as the earlier
  example, \texttt{H(\"a|ae?)ndel}.

  The precise syntax for regular expressions varies among tools and
  with context; more detail is given in the Chapter~\ref{chap:syntax}.
