\chapter{Formal language theory}
\label{chap:formal_lang_theory}

Regular expressions describe regular languages in formal language
theory. They have the same expressive power as regular grammars.

\section{Formal definition}
Regular expressions consist of constants and operator symbols that
denote sets of strings and operations over these sets,
respectively. The following definition is standard, and found as such
in most textbooks on formal language theory. Given a finite alphabet
$\Sigma$, the following constants are defined as regular expressions:
\begin{itemize}
\item (empty set) $\varnothing$ denoting the set $\varnothing$.
\item (empty string) $\varepsilon$ denoting the set containing only
  the "empty" string, which has no characters at all.
\item (literal character) a in $\Sigma$ denoting the set containing
  only the character a.
\end{itemize}
Given regular expressions $R$ and $S$, the following operations over them
are defined to produce regular expressions:
\begin{itemize}
\item (concatenation) $RS$ denotes the set of strings that can be
  obtained by concatenating a string in $R$ and a string in $S$. For
  example \{``ab'', ``c''\}\{``d'', ``ef''\} $=$ \{``abd'', ``abef'',
  ``cd'', ``cef''\}.
\item (alternation) $R | S$ denotes the set union of sets described by
  $R$ and $S$. For example, if $R$ describes \{``ab'', ``c''\} and $S$
  describes \{``ab'', ``d'', ``ef''\}, expression $R | S$ describes
  \{``ab'', ``c'', ``d'', ``ef''\}.
\item (Kleene star) $R*$ denotes the smallest superset of set
  sescribed by $R$ that contains $\varepsilon$ and is closed under
  string concatenation. This is the set of all strings that can be
  made by concatenating any finite number (including zero) of strings
  from set described by $R$. For example, \{``0'', ``1''\}* is the set
  of all finite binary strings (including the empty string), and
  \{``ab'', ``c''\}* $=$ \{$\varepsilon$, ``ab'', ``c'', ``abab'',
  ``abc'', ``cab'', ``cc'', ``ababab'', ``abcab'', \ldots\}.
\end{itemize}

To avoid parentheses it is assumed that the Kleene star has the
highest priority, then concatenation and then alternation. If there is
no ambiguity then parentheses may be omitted. For example,
\verb|(ab)c| can be written as \verb|abc|, and \verb/a|(b(c*))/ can be
written as \verb/a|bc*/. Many textbooks use the symbols $\cup$, $+$,
or $\vee$ for alternation instead of the vertical bar.

\textbf{Examples:}
\begin{itemize}
\item \verb/a|b*/ denotes \{$\varepsilon$, ``a'', ``b'', ``bb'',
  ``bbb'', \ldots\}
\item \verb/(a|b)*/ denotes the set of all strings with no symbols
  other than ``a'' and ``b'', including the empty string:
  \{$\varepsilon$, ``a'', ``b'', ``aa'', ``ab'', ``ba'', ``bb'',
  ``aaa'', \ldots\}
\item \verb/ab*(c|ε)/ denotes the set of strings starting with ``a'',
  then zero or more ``b''s and finally optionally a ``c'': \{``a'',
  ``ac'', ``ab'', ``abc'', ``abb'', ``abbc'', \ldots\}
\item \verb/(0|(1(01*0)*1))*/ denotes the set of binary numbers that
  are multiples of 3: \{$\varepsilon$, ``0'', ``00'', ``11'', ``000'',
  ``011'', ``110'', ``0000'', ``0011'', ``0110'', ``1001'', ``1100'',
  ``1111'', ``00000'', \ldots\}
\end{itemize}

\section{Expressive power and compactness}
The formal definition of regular expressions is purposely parsimonious
and avoids defining the redundant quantifiers $?$ and $+$, which can
be expressed as follows: $a+ = aa*$, and $a? = (a|\varepsilon)$.
Sometimes the complement operator is added, to give a
\textsl{generalized regular expression}; here $R^c$ matches all
strings over $\Sigma^*$ that do not match $R$. In principle, the
complement operator is redundant, as it can always be circumscribed by
using the other operators. However, the process for computing such a
representation is complex, and the result may require expressions of a
size that is double exponentially larger.

Regular expressions in this sense can express the regular languages,
exactly the class of languages accepted by deterministic finite
automata. There is, however, a significant difference in
compactness. Some classes of regular languages can only be described
by deterministic finite automata whose size grows exponentially in the
size of the shortest equivalent regular expressions. The standard
example here is the languages $L_k$ consisting of all strings over the
alphabet \{$a$, $b$\} whose $k^{th}$-from-last letter equals $a$. On
one hand, a regular expression describing $L_4$ is given by
\verb/(a|b)*a(a|b)(a|b)(a|b)/. Generalizing this pattern to $L_k$
gives the expression
%% (a|b)*a(a|b)(a|b)...(a|b)
%%       ------------------
%%             k-1 times
\[(a|b)^*\underbrace{a(a|b)(a|b)\ldots(a|b)}_{k-1~times}\]

On the other hand, it is known that every deterministic finite
automaton accepting the language $L_k$ must have at least $2^k$
states. Luckily, there is a simple mapping from regular expressions to
the more general nondeterministic finite automata (NFAs) that does not
lead to such a blowup in size; for this reason NFAs are often used as
alternative representations of regular languages. NFAs are a simple
variation of the type-3 grammars of the Chomsky hierarchy.

Finally, it is worth noting that many real-world ``regular expression''
engines implement features that cannot be described by the regular
expressions in the sense of formal language theory; see below for more
on this.

\section{Deciding equivalence of regular expressions}
As seen in many of the examples above, there is more than one way to
construct a regular expression to achieve the same results.

It is possible to write an algorithm that, for two given regular
expressions, decides whether the described languages are equal; the
algorithm reduces each expression to a minimal deterministic finite
state machine, and determinies whether they are isomorphic
(equivalent).

The redundancy can be eliminated by using Kleene star and set union to
find an interesting subset of regular expressions that is still fully
expressive, but perhaps their use can be restricted. This is a
surprisingly difficult problem. As simple as the regular expressions
are, there is no method to systematically rewrite them to some normal
form. The lack of axiom in the past led to the star height problem. In
1991, Dexter Kozen axiomatized regular expressions with Kleene
algebra; see Kleene algebra History for details.
