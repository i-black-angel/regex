\chapter{Syntax}
\label{chap:syntax}

A regexp \textsl{pattern} matches a target \textsl{string}. The
pattern is composed of a sequence of \textsl{atoms}. An atom is a
single point within the regexp pattern which it tries to match to the
target string. The simplest atom is a literal, but grouping parts of
the pattern to match an atom will require using \verb|()| as
metacharacters. Metacharacters help form: \textsl{atoms};
\textsl{quantifiers} telling how many atoms (and whether it is
\textsl{greedy} quantifier or not); a logical OR character, which
offers a set of alternatives, and a logical NOT character, which
negates an atom's existence; and back references to refer to previous
atoms of a completing pattern of atoms. A match is made, not when all
the atoms of the string are matched, but rather when all the pattern
atoms in the regular expression have matched. The idea is to make a
small pattern of characters stand for a large number of possible
strings, rather than compiling a large list of all the literal
possibilities.

Depending on the regexp processor there are about fourteen
metacharacters, characters that may or may not have their literal
character meaning, depending on context, or whether they are
``escapted'', i.e. preceded by an escape sequence, in this case, the
backslash \verb|\|. Modern and POSIX extended regular expressions use
metacharacters more often than their literal meaning, so to avoid
"backslash-osis" it makes sense to have a metacharacter escape to a
literal mode; but starting out, it makes more sense to have the four
bracketing metacharacters \verb|()| and \verb|{}| be primarily
literal, and "escape" this usual meaning to become
metacharacters. Common standards implement both. The usual
metacharacters are \verb/{}[]()^$.|*+?/ and \verb|\|. The usual
characters that become metacharacters when escaped are \verb|dsw.DSW|
and \verb|N|.

\section{Delimiters}
When entering a regular expression in a programming language, they may
be represented as a usual string literal, hence usually quoted; this
is common in C, Java, and Python for instance, where the regular
expression \verb|re| is entered as \verb|"re"|. However, they are
often written with slashes as delimiters, as in \verb|/re/| for the
regular expression \verb|re|. This originates in \verb|ed|, where
\verb|/| is the editor command for searching, and an expression
\verb|/re/| can be used to specify a range of lines (matching the
pattern), which can be combined with other commands on either side,
most famously \verb|g/re/p| as in \verb|grep| (``global regex
print''), which is included in most Unix-based operating systems, such
as Linux distributions. A similar convention is used in \verb|sed|,
where search and replace is given by \verb|s/regexp/replacement/| and
patterns can be joined with a comma to specify a range of lines as in
\verb|/rel/,/re2/|. This notation is particularly well-known due to
its use in Perl, where it forms part of the syntax distinct from
normal string literals. In some cases, such as \verb|sed| and Perl,
alternative delimiters can be used to avoid collision with contents,
and to avoid having to escape occurrences of the delimiter character
in the contents. For example, in \verb|sed| the command \verb|s,/,X,|
will replace a \verb|/| with an \verb|X|, using commas as delimiters.

\section{Standards}
The IEEE POSIX standard has three sets of compliance: BRE, ERE, and
SRE for basic, Extended, and Simple Regular Expressions. SRE is
deprecated, in favor of BRE, as both provide backward
compatibility. The subsection below covering the \textsl{character}
classes applies to both BRE and ERE.

BRE and ERE work together. ERE adds \verb|?|, \verb|+|, and \verb/|/,
and it removes the need to escape the metacharacters \verb|()| and
\verb|{}|, which are \textsl{required} in BRE. Furthermore, as long as
the POSIX standard syntax for regular expressions is adhered to, there
can be, and often is, additional syntax to serve specific (yet POSIX
compliant) applications. Although POSIX.2 leaves some implementation
specifics undefined, BRE and ERE provide a ``standard'' which has
since been adopted as the default syntax of many tools, where the
choice of BRE or ERE modes is usually a supported option. For example,
GNU \verb|grep| has the following options: \verb|"grep -E"| for ERE,
and \verb|"grep -G"| for BRE (the default), and \verb|"grep -P"| for
Perl regular expressions.

Perl regular expressions have become a de facto standard, having a
rich and powerful set of atomic expressions. Perl has no ``basic'' or
``extended'' levels, where the \verb|()| and \verb|{}| may or may not
have literal meanings. They are always metacharacters, as they are in
``extended'' mode for POSIX. To get their \textsl{literal} meaning,
you escape them. Other metacharacters are known to be literal or
symbolic based on context alone. Perl offers much more functionality:
``lazy'' regular expressions, backtracking, named capture groups, and
recursive patterns, all of which are powerful additions to POSIX
BRE/ERE. (See lazy quantification below.)

\section{POSIX basic and extended}
In the POSIX standard, Basic Regular Syntax, BRE, requires that the
metacharacters \verb|()| and \verb|{}| be designated \verb|\(\)| and 
\verb|\{\}|, whereas Extended Regular Syntax, ERE, does not.
