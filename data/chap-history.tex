\chapter{History}
\label{chap:history}

Regular expressions originated in 1956, when mathematician Stephen
cole Kleene described regular languages using his mathematical
notation called \textsl{regular} sets. These arose in theoretical
computer science, in the subfields of automata theory (models of
computation) and the description and classification of formal
languages. Other early implementations of pattern matching include the
SNOBOL language, which did not use regular expressions, but instead
its own syntax.

Regular expressions entered popular use from 1968 in two uses: pattern
matching in a text editor and lexical analysis in a compiler. Among
the first appearances of regular expressions in program form was when
Ken Thompson built Kleene's notation into the editor \verb|QED| as a
means to match patterns in text files. For speed, Thompson implemented
regular expression matching by just-in-time compilation (JIT) to IBM
7094 code on the Compatible Time-Sharing System, an important early
example of JIT compilation. He later added this capability to the Unix
editor \verb|ed|, which eventually led to the popular search tool
\verb|grep|'s use of regular expressions (``grep'' is a word derived
from the command for regular expression searching in the \verb|ed|
editor: \verb|g/re/p| meaning ``Global search for Regular Expression
and Print matching lines''). Around the same time when Thompson
developed \verb|QED|, a group of researchers including Douglas T. Ross
implemented a tool based on regular expressions that is used for
lexical analysis in compiler design.

Many variations of these original forms of regular expressions were
used in Unix programs at Bell Labs in the 1970s, including \verb|vi|,
\verb|lex|, \verb|sed|, \verb|AWK|, and \verb|expr|, and in other
programs such as \verb|Emacs|. Regular expressions were subsequently
adopted by a wide range of programs, with these early forms
standardized in the POSIX.2 standard in 1992.

In the 1980s more complicated regular expressions arose in Perl, which
originally derived from a regex library writtern by Henry Spencer
(1986), who later wrote an implementation of \textsl{Advanced Regular
  Expressions} for Tcl. The Tcl library is a hybrid NFA/DFA
implementation with improved performance characteristics, earning
praise from Jeffrey Friedl who said, ``\ldots it really seems quite
wonderful.'' Software projects that have adopted Spencer's Tcl regular
expression implementation include PostpreSQL. Perl later expanded on
Spencer's original library to add many new features, but has not yet
caught up with Spencer's Advanced Regular Expressions implementation
in terms of performance or Unicode handling. Part of the effort in the
design of Perl 6 is to improve Perl's regular expression integration,
and to increase their scope and capabilities to allow the definition
of parsing expression grammars. The result is a mini-language called
Perl 6 rules, which are used to define Perl 6 grammar as well as
provide a tool to programmers in the language. These rules maintain
existing features of Perl 5.x regular expressions, but also allow
BNF-style definition of a recursive descent parser via sub-rules.

The use of regular expressions in structured information standards for
document and database modeling started in the 1960s and expanded in
the 1980s when industry standards like ISO SGML (precursored by ANSI
``GCA 101-1983'') consolidated. The kernel of the structure
specification language standards consists of regular expressions. Its
use is evident in the DTD element group syntax.

Starting in 1997, Philip Hazel developed \verb|PCRE| (Perl Compatible
Regular Expressions), which attempts to closely mimic Perl's regular
expression functionality and is used by many modern tools including
PHP and Apache HTTP Server.

Today regular expressions are widely supported in programming
languages, text processing programs (particular lexers), advanced text
editors, and some other programs. Regular expression support is part
of the standard library of many programming languages, including Java
and Python, and is built into the syntax of others, including Perl and
ECMAScript. Implementations of regular expression functionality is
often called a \textbf{regular expression engine}, and a number of
libraries are available for reuse.
